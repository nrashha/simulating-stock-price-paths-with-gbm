% Options for packages loaded elsewhere
% Options for packages loaded elsewhere
\PassOptionsToPackage{unicode}{hyperref}
\PassOptionsToPackage{hyphens}{url}
\PassOptionsToPackage{dvipsnames,svgnames,x11names}{xcolor}
%
\documentclass[
  12pt]{article}
\usepackage{xcolor}
\usepackage{amsmath,amssymb}
\setcounter{secnumdepth}{5}
\usepackage{iftex}
\ifPDFTeX
  \usepackage[T1]{fontenc}
  \usepackage[utf8]{inputenc}
  \usepackage{textcomp} % provide euro and other symbols
\else % if luatex or xetex
  \usepackage{unicode-math} % this also loads fontspec
  \defaultfontfeatures{Scale=MatchLowercase}
  \defaultfontfeatures[\rmfamily]{Ligatures=TeX,Scale=1}
\fi
\usepackage{lmodern}
\ifPDFTeX\else
  % xetex/luatex font selection
\fi
% Use upquote if available, for straight quotes in verbatim environments
\IfFileExists{upquote.sty}{\usepackage{upquote}}{}
\IfFileExists{microtype.sty}{% use microtype if available
  \usepackage[]{microtype}
  \UseMicrotypeSet[protrusion]{basicmath} % disable protrusion for tt fonts
}{}
\makeatletter
\@ifundefined{KOMAClassName}{% if non-KOMA class
  \IfFileExists{parskip.sty}{%
    \usepackage{parskip}
  }{% else
    \setlength{\parindent}{0pt}
    \setlength{\parskip}{6pt plus 2pt minus 1pt}}
}{% if KOMA class
  \KOMAoptions{parskip=half}}
\makeatother
% Make \paragraph and \subparagraph free-standing
\makeatletter
\ifx\paragraph\undefined\else
  \let\oldparagraph\paragraph
  \renewcommand{\paragraph}{
    \@ifstar
      \xxxParagraphStar
      \xxxParagraphNoStar
  }
  \newcommand{\xxxParagraphStar}[1]{\oldparagraph*{#1}\mbox{}}
  \newcommand{\xxxParagraphNoStar}[1]{\oldparagraph{#1}\mbox{}}
\fi
\ifx\subparagraph\undefined\else
  \let\oldsubparagraph\subparagraph
  \renewcommand{\subparagraph}{
    \@ifstar
      \xxxSubParagraphStar
      \xxxSubParagraphNoStar
  }
  \newcommand{\xxxSubParagraphStar}[1]{\oldsubparagraph*{#1}\mbox{}}
  \newcommand{\xxxSubParagraphNoStar}[1]{\oldsubparagraph{#1}\mbox{}}
\fi
\makeatother

\usepackage{color}
\usepackage{fancyvrb}
\newcommand{\VerbBar}{|}
\newcommand{\VERB}{\Verb[commandchars=\\\{\}]}
\DefineVerbatimEnvironment{Highlighting}{Verbatim}{commandchars=\\\{\}}
% Add ',fontsize=\small' for more characters per line
\usepackage{framed}
\definecolor{shadecolor}{RGB}{241,243,245}
\newenvironment{Shaded}{\begin{snugshade}}{\end{snugshade}}
\newcommand{\AlertTok}[1]{\textcolor[rgb]{0.68,0.00,0.00}{#1}}
\newcommand{\AnnotationTok}[1]{\textcolor[rgb]{0.37,0.37,0.37}{#1}}
\newcommand{\AttributeTok}[1]{\textcolor[rgb]{0.40,0.45,0.13}{#1}}
\newcommand{\BaseNTok}[1]{\textcolor[rgb]{0.68,0.00,0.00}{#1}}
\newcommand{\BuiltInTok}[1]{\textcolor[rgb]{0.00,0.23,0.31}{#1}}
\newcommand{\CharTok}[1]{\textcolor[rgb]{0.13,0.47,0.30}{#1}}
\newcommand{\CommentTok}[1]{\textcolor[rgb]{0.37,0.37,0.37}{#1}}
\newcommand{\CommentVarTok}[1]{\textcolor[rgb]{0.37,0.37,0.37}{\textit{#1}}}
\newcommand{\ConstantTok}[1]{\textcolor[rgb]{0.56,0.35,0.01}{#1}}
\newcommand{\ControlFlowTok}[1]{\textcolor[rgb]{0.00,0.23,0.31}{\textbf{#1}}}
\newcommand{\DataTypeTok}[1]{\textcolor[rgb]{0.68,0.00,0.00}{#1}}
\newcommand{\DecValTok}[1]{\textcolor[rgb]{0.68,0.00,0.00}{#1}}
\newcommand{\DocumentationTok}[1]{\textcolor[rgb]{0.37,0.37,0.37}{\textit{#1}}}
\newcommand{\ErrorTok}[1]{\textcolor[rgb]{0.68,0.00,0.00}{#1}}
\newcommand{\ExtensionTok}[1]{\textcolor[rgb]{0.00,0.23,0.31}{#1}}
\newcommand{\FloatTok}[1]{\textcolor[rgb]{0.68,0.00,0.00}{#1}}
\newcommand{\FunctionTok}[1]{\textcolor[rgb]{0.28,0.35,0.67}{#1}}
\newcommand{\ImportTok}[1]{\textcolor[rgb]{0.00,0.46,0.62}{#1}}
\newcommand{\InformationTok}[1]{\textcolor[rgb]{0.37,0.37,0.37}{#1}}
\newcommand{\KeywordTok}[1]{\textcolor[rgb]{0.00,0.23,0.31}{\textbf{#1}}}
\newcommand{\NormalTok}[1]{\textcolor[rgb]{0.00,0.23,0.31}{#1}}
\newcommand{\OperatorTok}[1]{\textcolor[rgb]{0.37,0.37,0.37}{#1}}
\newcommand{\OtherTok}[1]{\textcolor[rgb]{0.00,0.23,0.31}{#1}}
\newcommand{\PreprocessorTok}[1]{\textcolor[rgb]{0.68,0.00,0.00}{#1}}
\newcommand{\RegionMarkerTok}[1]{\textcolor[rgb]{0.00,0.23,0.31}{#1}}
\newcommand{\SpecialCharTok}[1]{\textcolor[rgb]{0.37,0.37,0.37}{#1}}
\newcommand{\SpecialStringTok}[1]{\textcolor[rgb]{0.13,0.47,0.30}{#1}}
\newcommand{\StringTok}[1]{\textcolor[rgb]{0.13,0.47,0.30}{#1}}
\newcommand{\VariableTok}[1]{\textcolor[rgb]{0.07,0.07,0.07}{#1}}
\newcommand{\VerbatimStringTok}[1]{\textcolor[rgb]{0.13,0.47,0.30}{#1}}
\newcommand{\WarningTok}[1]{\textcolor[rgb]{0.37,0.37,0.37}{\textit{#1}}}

\usepackage{longtable,booktabs,array}
\usepackage{calc} % for calculating minipage widths
% Correct order of tables after \paragraph or \subparagraph
\usepackage{etoolbox}
\makeatletter
\patchcmd\longtable{\par}{\if@noskipsec\mbox{}\fi\par}{}{}
\makeatother
% Allow footnotes in longtable head/foot
\IfFileExists{footnotehyper.sty}{\usepackage{footnotehyper}}{\usepackage{footnote}}
\makesavenoteenv{longtable}
\usepackage{graphicx}
\makeatletter
\newsavebox\pandoc@box
\newcommand*\pandocbounded[1]{% scales image to fit in text height/width
  \sbox\pandoc@box{#1}%
  \Gscale@div\@tempa{\textheight}{\dimexpr\ht\pandoc@box+\dp\pandoc@box\relax}%
  \Gscale@div\@tempb{\linewidth}{\wd\pandoc@box}%
  \ifdim\@tempb\p@<\@tempa\p@\let\@tempa\@tempb\fi% select the smaller of both
  \ifdim\@tempa\p@<\p@\scalebox{\@tempa}{\usebox\pandoc@box}%
  \else\usebox{\pandoc@box}%
  \fi%
}
% Set default figure placement to htbp
\def\fps@figure{htbp}
\makeatother





\setlength{\emergencystretch}{3em} % prevent overfull lines

\providecommand{\tightlist}{%
  \setlength{\itemsep}{0pt}\setlength{\parskip}{0pt}}



 
\usepackage[]{natbib}
\bibliographystyle{agsm}


\addtolength{\oddsidemargin}{-.5in}%
\addtolength{\evensidemargin}{-1in}%
\addtolength{\textwidth}{1in}%
\addtolength{\textheight}{1.7in}%
\addtolength{\topmargin}{-1in}%
\usepackage{booktabs}
\usepackage{longtable}
\usepackage{array}
\usepackage{multirow}
\usepackage{wrapfig}
\usepackage{float}
\usepackage{colortbl}
\usepackage{pdflscape}
\usepackage{tabu}
\usepackage{threeparttable}
\usepackage{threeparttablex}
\usepackage[normalem]{ulem}
\usepackage{makecell}
\usepackage{xcolor}
\makeatletter
\@ifpackageloaded{caption}{}{\usepackage{caption}}
\AtBeginDocument{%
\ifdefined\contentsname
  \renewcommand*\contentsname{Table of contents}
\else
  \newcommand\contentsname{Table of contents}
\fi
\ifdefined\listfigurename
  \renewcommand*\listfigurename{List of Figures}
\else
  \newcommand\listfigurename{List of Figures}
\fi
\ifdefined\listtablename
  \renewcommand*\listtablename{List of Tables}
\else
  \newcommand\listtablename{List of Tables}
\fi
\ifdefined\figurename
  \renewcommand*\figurename{Figure}
\else
  \newcommand\figurename{Figure}
\fi
\ifdefined\tablename
  \renewcommand*\tablename{Table}
\else
  \newcommand\tablename{Table}
\fi
}
\@ifpackageloaded{float}{}{\usepackage{float}}
\floatstyle{ruled}
\@ifundefined{c@chapter}{\newfloat{codelisting}{h}{lop}}{\newfloat{codelisting}{h}{lop}[chapter]}
\floatname{codelisting}{Listing}
\newcommand*\listoflistings{\listof{codelisting}{List of Listings}}
\makeatother
\makeatletter
\makeatother
\makeatletter
\@ifpackageloaded{caption}{}{\usepackage{caption}}
\@ifpackageloaded{subcaption}{}{\usepackage{subcaption}}
\makeatother
\usepackage{bookmark}
\IfFileExists{xurl.sty}{\usepackage{xurl}}{} % add URL line breaks if available
\urlstyle{same}
\hypersetup{
  pdftitle={Geometric Brownian Motion in modelling Options and Stock Prices},
  pdfauthor={Nahian Rashha},
  pdfkeywords={stochastic processes, Markov property, financial
modelling, estimation, Black-Scholes framework},
  colorlinks=true,
  linkcolor={blue},
  filecolor={Maroon},
  citecolor={Blue},
  urlcolor={Blue},
  pdfcreator={LaTeX via pandoc}}



\begin{document}


\def\spacingset#1{\renewcommand{\baselinestretch}%
{#1}\small\normalsize} \spacingset{1}


%%%%%%%%%%%%%%%%%%%%%%%%%%%%%%%%%%%%%%%%%%%%%%%%%%%%%%%%%%%%%%%%%%%%%%%%%%%%%%

\date{December 15, 2025}
\title{\bf Geometric Brownian Motion in modelling Options and Stock
Prices}
\author{
Nahian Rashha\thanks{The author gratefully acknowledges her sister and
friends for their encouragement and support, as well as her professor
for guidance and feedback throughout the development of this project.}\\
Department of Statistics, Amherst College\\
}
\maketitle

\bigskip
\bigskip
\begin{abstract}
This paper provides an expository review of geometric Brownian motion
(GBM) as a stochastic model for asset prices and its role in the
Black--Scholes framework for pricing options. Beginning with
foundational concepts from probability theory and stochastic processes,
the paper develops the mathematical structure of Brownian motion,
introduces GBM through stochastic differential equations, and explains
how the lognormal price model leads to the Black--Scholes formula under
ideal market assumptions. A simulation study then illustrates how the
approximated price of a hypothetical European call option under GBM,
computed by averaging discounted payoffs, is consistent with the
Black--Scholes closed-form solution. Finally, the model is applied to
historical S\&P 500 data to estimate volatility, simulate future index
paths, and price a hypothetical European call option. While the
empirical application highlights certain limitations of the GBM
assumptions in real markets, it also illustrates the practical
usefulness of GBM as an approximation method for asset prices due to its
relative success in capturing market behavior over other mathematical
models.
\end{abstract}

\noindent%
{\it Keywords:} stochastic processes, Markov property, financial
modelling, estimation, Black-Scholes framework
\vfill

\newpage
\spacingset{1.9} % DON'T change the spacing!


\section{Introduction}\label{sec-intro}

In the financial realm, options are defined as a financial instrument or
legal contract that gives its owner the right to either buy or sell a
quantity of assets, typically stocks, at a predetermined price, known as
a strike price, at or within a certain date, called an expiration date
\cite{fidelityoptions}. Options that can be exercised at any time within
the expiration date are termed as `American' while those that can only
be exercised on the expiration date are referred to as `European.'
Options that give its owner the right to buy the underlying stocks
within the contract are `call' options, and those that come with a legal
right to sell the contract's underlying assets are called `put' options
\cite{fidelityoptions}.

In general, it is safe to assume that ``the higher the price of the
stock, the greater the value of the option,'' \cite{blackscholes1973}
and therefore, accurately forecasting the behavior of the underlying
assets or stocks of an option is critical for a trader to understand
what kind of options-based trading strategy might be most profitable for
them to pursue. For instance, if a stock price can be forecasted to be
much greater than its strike price negotiated in a call option on or
within its expiration date, then it is profitable for the trader to buy
that call option and then, if the forecast holds, the option will most
likely be exercised by the trader \cite{blackscholes1973}. Therefore,
traders, risk managers, and financial analysts rely on statistical
models to forecast asset price movements, assess risk, and determine
whether an option is ``fairly priced.''

However, modeling stock prices deterministically is challenging since in
real markets, prices fluctuate continuously, respond to unpredictable
information, and exhibit randomness. Financial analysts assume that the
stock market operates under the efficient market hypothesis (EMH), which
states that given a particular information set, the market already
captures all of that information and the prices would be unaffected if
the information was fully revealed to market participants
\cite{sewell2012}. So, early work in options pricing posit that asset
prices evolve in a time-dependent and uncertain manner and require
stochastic modelling assumptions \cite{samuelson1965}. The most
prominent model to forecast options price is Geometric Brownian Motion
(GBM), which is a ``continuous-time stochastic process,''
\cite{entropygbm} where the logarithm of the stock prices follows a
Brownian motion with drift. This theory was central to the development
of the `Black-Scholes' formula, which prices European options under the
assumption that the underlying asset follows a lognormal distribution
and therefore, can be modeled using a GBM framework
\cite{blackscholes1973}. In this expository review, I will introduce the
necessary background for stochastic processes, develop the mathematical
intuition behind GBM, and explain how the Black--Scholes framework links
GBM to option valuation. I will then briefly explore prominent examples
of GBM applied to financial problems in literature. Finally, I will
build a demonstration model that applies GBM and the Black--Scholes
formula to real stock-market data.

\section{An Expository Review of Geometric Brownian Motion
(GBM)}\label{sec-exp}

\subsection{Introduction to Stochastic Processes}\label{sec-exp-stoch}

We recall the definition of a probability space as the triple
\((\Omega, \mathcal{F}, \mathbb{P})\) where,

\begin{enumerate}
\def\labelenumi{\arabic{enumi}.}
\item
  \(\Omega\) is the sample space, representing the set of all possible
  outcomes of the random experiment under consideration.
\item
  \(\mathcal{F}\) is a collection of subsets of \(\Omega\), called
  \emph{events}, to which probabilities can be assigned. This collection
  is chosen so that it contains all events of interest and is closed
  under complements and countable unions. Intuitively, \(\mathcal{F}\)
  specifies those events that are observable and meaningful.
\item
  \(\mathbb{P}\) is a probability measure defined on the domain
  \(\mathcal{F}\), satisfying
\end{enumerate}

\begin{equation}
\mathbb{P} : \mathcal{F} \to [0,1], \qquad \mathbb{P}(\Omega)=1
\label{eq:prob_measure}
\end{equation}

and countable additivity:

\begin{equation}
\mathbb{P}\!\left(\bigcup_{i=1}^{\infty} A_i\right)
= \sum_{i=1}^{\infty} \mathbb{P}(A_i),
\qquad \text{for all disjoint } A_i \in \mathcal{F}.
\label{eq:countable_additivity}
\end{equation}

\cite{pettersdong}

A stochastic (or random) process is a collection of random variables
defined on a given probability space
\((\Omega, \mathcal{F}, \mathbb{P})\). Each of these random variables
represent the same underlying quantity, but capture observations of the
quantity at different points in time. Taken together, the stochastic
process models the evolution of that uncertain quantity over time
\cite{pettersdong}. Formally, following Petters and Dong, given a
probability space \((\Omega, \mathcal{F}, \mathbb{P})\), a stochastic
process is defined as

\begin{equation}
\{X(t): t \in J\},
\label{eq:stochastic_process}
\end{equation}

where each \(X(t)\) is a real-valued random variable defined on the
common probability space and \(J\), a non-empty subset of the reals, is
an index set of the process \(X\) representing time \cite{pettersdong}.
If \(J=\mathbb{N}\), then the random process evolves over discrete time
and if \(J=[0,\infty)\), it evolves over continuous time
\cite{pettersdong}.

As discussed by Samuelson in his foundational 1965 paper, asset prices
evolve over time in a way that incorporates both deterministic trends
and random fluctuations, motivating the use of continuous-time
stochastic models when forecasting stock prices \cite{samuelson1965}.

\subsection{Markov Processes and the Markov
Property}\label{sec-exp-markov}

In this section, we review necessary definitions for a widely used
stochastic process fundamental to understanding the theory for GBM.
First, we recall that a set \(S\) is called the \emph{state space} of a
system or process if it consists of all possible values that the random
variables associated with that system may take \cite{hoelportstone}.
This state space is called discrete if \(S\) is a finite or countable
set, and \emph{continuous} if it is an uncountable subset of
\(\mathbb{R}^d\), an interval of real numbers.. For the purposes of
illustration, we first restrict attention to stochastic processes with
discrete state spaces. This restriction will be relaxed later when we
introduce continuous-state Markov processes such as Brownian motion and
GBM.

Now, let, \(\{X_n : n \in J\}\) where \(J = \{0,1,2,3,\ldots\}\) be a
stochastic process on a probability space
\((\Omega,\mathcal{F},\mathbb{P})\) with a discrete state space \(S\),
as defined above. Following Hoel, Port, and Stone, this stochastic
process is called a \emph{Markov process} if, for all states
\(x_0, x_1, \ldots, x_{n-1}, x_{n}, x_{n+1} \in S\), the transition
probabilities satisfy the following \emph{Markov rule}:

\begin{equation}
\mathbb{P}(X_{n+1} = x_{n+1} \mid 
    X_0 = x_{0},\ldots,
    X_{n-1} = x_{n-1}, X_{n} = x_{n} )
\;=\;
\mathbb{P}(X_{n+1} = x_{n+1} \mid X_n = x_{n}),
\label{eq:markov_property}
\end{equation}

whenever these conditional probabilities are well-defined
\cite{hoelportstone}. Intuitively, the Markov rule formalizes the idea
that the future evolution of a Markov process depends only on its
present state and not on past states. Once the current state
\(X_n = x_n\) is known, any additional information about earlier states
\(X_0, \ldots, X_{n-1}\) provides no further predictive power for
determining the distribution of \(X_{n+1}\) \cite{hoelportstone}. For
our purposes, Markov processes with discrete state spaces will be
referred to as \emph{Markov chains} as per Hoel, Port, and Stone
\cite{hoelportstone}, while the focus of this paper is on Markov
processes with continuous state spaces, which we will simply refer to as
\emph{Markov processes}.

For a Markov process, the conditional rules

\begin{equation}
\mathbb{P}(X_{n+1} = y \mid X_n = x),
\label{eq:transition_probability}
\end{equation}

are called the \emph{transition probabilities} to `go' from state \(x\)
at time \(n\) to state \(y\) at time \(n+1\). These transition
probabilities are called `stationary' if they are independent of \(n\)
\cite{hoelportstone}. Generalizing this definition, if
\(\{X_n : n \ge 0\}\) is a Markov process with state space \(S\), then
for \(x, y \in S\), the function

\begin{equation}
P(x,y) = \mathbb{P}(X_1 = y,\, X_0 = x), \qquad x, y \in S,
\label{eq:joint_transition_prob}
\end{equation}

is called the `transition function' of the chain which satisfies the
following:

\begin{equation}
P(x,y) \ge 0 \quad \text{for all } x,y \in S,
\qquad \text{and} \qquad
\sum_{y \in S} P(x,y) = 1 \quad \text{for all } x \in S.
\label{eq:transition_matrix_properties}
\end{equation}

These are also known as `one-step' transition probabilities of the
Markov chain \cite{hoelportstone}. Finally, the function

\begin{equation}
\pi_0(x) = \mathbb{P}(X_0 = x), \qquad x \in S,
\label{eq:initial_distribution}
\end{equation}

is called the initial distribution of the Markov chain such that the
following hold:

\begin{equation}
\pi_0(x) \ge 0 \quad \text{for all } x \in S,
\qquad \text{and} \qquad
\sum_{x \in S} \pi_0(x) = 1,
\label{eq:initial_distribution_properties}
\end{equation}

\cite{hoelportstone}.

In continuous-state Markov processes, transition probabilities are
described by probability density functions rather than discrete
transition probabilities, and summations over states are replaced by
integrals \cite{hoelportstone}.

Both Brownian motion, which underlies geometric Brownian motion (GBM),
and GBM itself are continuous-time Markov processes and therefore,
satisfy the Markov property \cite{pettersdong}. The relevance of the
Markov property to financial modeling lies in its role as a foundational
assumption for continuous-time asset price dynamics. In particular, if
the price of a common stock is modeled using GBM, the Markov property
implies that the conditional distribution of the stock's future prices
depends only on the current price and not on the history of past prices
\cite{samuelson1965}.

\subsection{Illustrative Example: The Weather Chain}\label{sec-exp-ex}

\begin{figure}[H]

\centering{

\includegraphics[width=0.7\linewidth,height=\textheight,keepaspectratio]{mc.png}

}

\caption{\label{fig-mc}An Illustrative Markov Chain: The Weather Chain.}

\end{figure}%

Here, we introduce a simple illustrative example of a \emph{Markov
chain} to solidify understanding of the Markov property before
transitioning to continuous-time Markov processes. Figure~\ref{fig-mc}
illustrates a simple discrete-time Markov chain with state space

\begin{equation}
S = \{\text{Rainy}, \text{Cloudy}, \text{Sunny}\},
\label{eq:state_space_example}
\end{equation}

which models the evolution of daily weather conditions. At each time
step, exactly one state is occupied, and transitions to a new state at
the next time step occur according to the labeled transition
probabilities \cite{hoelportstone}, \cite{geeksforgeeks_markov_nlp}.

The arrows indicate possible one-step transitions between weather
states, while the numerical labels specify the corresponding transition
probabilities. For example, if the current state is Rainy, the chain
remains Rainy with probability \(0.5\), transitions to Cloudy with
probability \(0.3\), and transitions to Sunny with probability \(0.2\).
Similarly, if the current state is Sunny, the chain remains Sunny with
probability \(0.8\).

This diagram provides a concrete illustration of the Markov property.
The probability distribution of tomorrow's weather depends only on
today's weather and not on the sequence of weather outcomes observed on
previous days. Once the current state is known, all relevant information
needed to determine the distribution of the next state is fully captured
\cite{hoelportstone}.

\subsection{Brownian Motion}\label{sec-exp-brown}

We now introduce the continuous-time stochastic process `Brownian
Motion', which serves as the foundational definition for GBM. Brownian
motion, also known as a \emph{Wiener process}, is the classical model
for the continuous random fluctuations and the highly irregular motion
observed in microscopic particles suspended in liquid. Following the
standard definition, a stochastic process \(\{B(t) : t \ge 0\}\) is
called a \emph{standard Brownian motion} if it satisfies the following
properties:

\begin{enumerate}
\def\labelenumi{\arabic{enumi}.}
\item
  \(B(0) = 0\)
\item
  For all \(0 \le s < t\),

  \[
  B(t) - B(s) \sim \mathcal{N}(0, t-s). 
  \]
\item
  For any \(0 \le t_1 < t_2 < \cdots < t_n\), the random variables

  \[
  B(t_2) - B(t_1),\;
  B(t_3) - B(t_2),\;\ldots,\;
  B(t_n) - B(t_{n-1})
  \]

  are independent \cite{hoelportstone}, \cite{pettersdong}.
\end{enumerate}

Using these defining properties of Brownian motion, we make comments on
some important results related to the process below:

\subsubsection{Mean and Variance of Brownian
Motion}\label{sec-exp-brown-1}

From the above definition, particularly from property 2, it immediately
follows that Brownian motion has mean zero at all times,

\begin{equation}
\mathbb{E}[B(t)] = 0, \qquad t \ge 0,
\label{eq:brownian_mean}
\end{equation}

and variance grows linearly in time,

\begin{equation}
\mathrm{Var}(B(t)) = t, \qquad t \ge 0.
\label{eq:brownian_variance}
\end{equation}

\subsubsection{Brownian Motion as a Gaussian Process and its Covariance
Function}\label{sec-exp-brown-2}

Brownian Motion \(\{B(t) : t \ge 0\}\) is also a Gaussian process, which
means that every finite linear combination of the random variables
\(B(t)\), for all \(t \in T\), is normally distributed
\cite{hoelportstone}. The defining properties of Brownian motion also
allow us to mathematically derive the covariance function for two random
variables \(B(t)\) and \(B(s)\), which is given by

\begin{equation}
\mathrm{Cov}(B(s), B(t)) = \min(s,t), \qquad s,t \ge 0.
\label{eq:brownian_covariance}
\end{equation}

A proof of this result is provided in Section~\ref{sec-appix}.

For Gaussian processes, if two random variables have the same mean and
covariance functions, as above, ``they also have the same joint
distribution functions,'' \cite{hoelportstone}. So, the mean and
covariance functions completely determine all finite-dimensional
distributions and fully characterize a Brownian motion process
\cite{hoelportstone}.

\subsubsection{Brownian Motion as a Markov
Process}\label{sec-exp-brown-3}

An important consequence of the defining properties of Brownian motion,
especially property 3, is that it satisfies the Markov property
\cite{pettersdong}. Intuitively, given the current value \(B(t)\), the
future increment \(B(t+h) - B(t)\) is independent of the past history of
the process prior to time \(t\), and depends only on the length of the
time interval \(h\) \cite{pettersdong}. As a result, once the present
state is known, past values provide no additional information for
predicting future behavior for a system modeled by this process
\cite{hoelportstone,pettersdong}.

\vspace{0.5em}
\vspace{0.5em}

While L.\textasciitilde Bachelier's seminal work established Brownian
motion as a mathematical foundation for modeling the continuous random
fluctuations of stock prices and is widely regarded as the birth of
mathematical finance \cite{bachelier1900_rhm2018}, it leads to the
result that in the long run, an option ``will increase in price
indefinitely, coming even to exceed the price of the common stock
itself,'' \cite{samuelson1965}. This outcome is counterintuitive, since
ownership of the stock is economically equivalent to holding a perpetual
option exercisable at zero cost \cite{samuelson1965}.

As Samuelson later emphasized, this anomaly arises because modeling
stock prices directly using Brownian motion allows prices to take on
negative values with positive probability \cite{samuelson1965}. Such
behavior is incompatible with financial reality, where stock prices are
strictly positive \cite{samuelson1965}. These considerations motivate
modeling the logarithm of the asset price as a Brownian motion, leading
to the geometric Brownian motion (GBM) model. By exponentiating Brownian
motion, GBM preserves the Markov property and yields lognormally
distributed stock prices that takes on strictly positice values, a
feature consistent with observed market dynamics and foundational to the
Black--Scholes framework
\cite{samuelson1965,blackscholes1973,pettersdong}. We formally develop
the GBM model as well as the Black--Scholes framework in the following
sections.

\subsection{Stochastic Differential Equations (SDE) and It\^{}o's
Lemma}\label{sec-exp-sde}

In modeling continuously evolving random systems, it is useful to
separate systematic trends from random fluctuations. On that note, we
present a parameter \(\mu\), representing the \emph{drift} of Brownian
motion, capturing its deterministic rate of change over time, and a
second parameter \(\sigma\), representing the \emph{volatility}, which
controls the magnitude of random fluctuations in the process. In
financial applications, \(\mu\) reflects the expected rate of return of
an asset such as a stock, while \(\sigma\) measures the uncertainty or
risk associated with its price movements \cite{pettersdong}.

Now, a stochastic process \(\{X(t)\}\) is called a Brownian motion with
drift \(\mu\) and scaling \(\sigma\) if it satisfies the stochastic
differential equation

\begin{equation}
dX(t) = \mu\,dt + \sigma\,dB(t),
\label{eq:arithmetic_brownian_motion}
\end{equation}

where \(\{B(t)\}\) is a standard Brownian motion \cite{pettersdong}. It
is possible to show that integrating this equation yields the explicit
representation

\begin{equation}
X(t) = x_0 + \mu t + \sigma B(t),
\label{eq:abm_solution}
\end{equation}

for some initial value \(x_0\in\mathbb{R}\) \cite{pettersdong}. Thus,
\{X(t)\} is a Brownian motion with drift \(\mu\) and scaling \(\sigma\)
if and only if it satisfies equation \ref{eq:abm_solution} for some
initial condition \(x_0\in\mathbb{R}\).

Let us suppose that the increment of a stochastic process \(X(t)\) over
a small time interval \([t,t+\Delta t]\) can be approximated by

\begin{equation}
\Delta X(t) \approx \mu(X(t),t)\,\Delta t + \sigma(t)\,\Delta B(t),
\label{eq:stochastic_increment}
\end{equation}

where \(\Delta B(t)\) denotes the increment of a Brownian motion, and
\(\mu\) and \(\sigma\) are as defined above \cite{pettersdong}. In
differential form, this relationship is written as the stochastic
differential equation

\begin{equation}
dX(t) = \mu(X(t),t)\,dt + \sigma(t)\,dB(t).
\label{eq:general_sde}
\end{equation}

This expression separates the evolution of \(X(t)\) into a deterministic
component, governed by the drift term \(\mu\), and a random noise
component \(\sigma\) driven by Brownian motion \cite{pettersdong}. The
differential equation itself provides a mathematical framework for
modeling the rate of change in Brownian motion, which is continuously
evolving process.

To analyze functions of stochastic processes that satisfy stochastic
differential equations, we use It\^{}o's formula, which plays the role
of the chain rule in stochastic calculus and provides us with a
framemwork to obtain solutions of s.d.e's. Let us suppose that \(X(t)\)
is a stochastic process that satisfies the stochastic differential
equation above, and let \(Y(t)=f(X(t),t)\), where \(f\) has continuous
first and second partial derivatives. Then It\^{}o's formula states that

\begin{equation}
dY
=
\left(
    f_t + \mu f_x + \tfrac{1}{2}\sigma^2 f_{xx}
\right)dt
+
\sigma f_x\,dB(t),
\label{eq:ito_formula}
\end{equation}

where subscripts denote partial derivatives \cite{pettersdong}.

In the next section, we define geometric Brownian motion (GBM) as the
solution to a stochastic differential equation and examine the
application of It\^{}o's formula to analyze functions of the GBM process
and derive the s.d.e.'s explicit solution \cite{pettersdong}.

\subsection{Geometric Brownian Motion}\label{sec-exp-geom}

We now introduce Geometric Brownian Motion (GBM), the stochastic process
used to model asset prices in modern-day financial mathematics. A
stochastic process \(\{X(t): t \ge 0\}\) is called a \emph{Geometric
Brownian Motion (GBM)} if it satisfies the It\^{}o stochastic
differential equation \ref{eq:general_sde} \cite{pettersdong}. It can be
shown that solving this equation with initial condition \(X(0)=x_0>0\),
where \(x_0 > 0\) by applying It\^{}o's formula with \(f(x) = ln(x)\)
yields the explicit solution

\begin{equation}
X(t)=x_0\exp\!\left[\left(\mu-\tfrac12\sigma^2\right)t+\sigma B(t)\right],
\label{eq:gbm_solution}
\end{equation}

\cite{pettersdong}.

Thus, \(\{X(t)\}\) with \(X(t) = exp[Y(t)]\) is a geometric Brownian
motion if and only if the corresponding log-process \(\{Y(t)\}\) is a
Brownian motion with drift and scaling \cite{pettersdong}.

In financial applications, \(X(t)\) represents the price of an asset,
most commonly a stock, and we therefore write \(S(t)\) in place of
\(X(t)\) for cleaner notation. We define the log-return process by

\begin{equation}
X(t) := \ln\!\left(\frac{S(t)}{S(0)}\right).
\label{eq:log_price_process}
\end{equation}

From the explicit solution of the GBM stochastic differential equation,
we see that since \(B(t)\) is normally distributed with mean \(0\) and
variance \(t\) \cite{pettersdong}, \(X(t)\) as defined as the log-return
process is then normally distributed with mean \(\mu t\) and variance
\(\sigma^2 t\) \cite{pettersdong}. Consequently, the stock price process
\(\{S(t): t \ge 0\}\) defined by

\begin{equation}
S(t) = S(0)e^{X(t)} = S(0)e^{\mu t + \sigma B(t)}, \qquad t \ge 0,
\label{eq:stock_price_from_log_process}
\end{equation}

is a geometric Brownian motion with parameters \(\mu\) and \(\sigma\)
\cite{pettersdong}. Finally, because \(X(t)\) is normally distributed,
the random variable \(S(t)\) follows a \emph{lognormal distribution},
written

\begin{equation}
S(t) \sim \text{lognormal}(\mu t, \sigma^2 t),
\label{eq:lognormal_distribution_stock}
\end{equation}

as shown in \cite{pettersdong}.

Geometric Brownian motion, as defined above, is a more appropriate model
for stock prices than standard Brownian motion, since stock prices are
strictly nonnegative and exhibit fluctuations that scale proportionally
with their current level \cite{pettersdong}. Clearly, when modeling
stock prices under GBM, the prices themselves are lognormally
distributed and take on strictly non-negative values. This property make
GBM the natural foundation for the Black--Scholes option pricing
framework, which we develop in the next section.

\subsection{The Black--Scholes Options Pricing
Framework}\label{sec-exp-bsm}

Before presenting the Black--Scholes formula, we briefly review the
structure of a \emph{European call option}. A European call option gives
its holder the right, but not the obligation, to purchase an underlying
asset at a fixed price \(K\), called the \emph{strike price}, at a fixed
future time \(T\), called the \emph{maturity} \cite{blackscholes1973}.
The payoff of the option at time \(T\) is

\begin{equation}
\max(S(T) - K,\, 0),
\label{eq:call_payoff}
\end{equation}

where \(S(T)\) denotes the price of the asset at maturity. This payoff
depends only on the value of the asset at time \(T\) and not on how the
price evolved over time \cite{pettersdong}.

In Section~\ref{sec-exp-geom}, we modeled the asset price, or the price
of a stock \(\{S(t)\}\) using geometric Brownian motion (GBM). Under
this model, \(S(T)\) is lognormally distributed, and therefore the
uncertainty in the option's payoff arises entirely from the randomness
in the terminal stock price \(\{S(t)\}\). Thus, once the parameters
\(S(0)\), \(\mu\), \(\sigma\), and \(T\) are specified, the distribution
of \(S(T)\) is fully determined \cite{pettersdong}.

Under certain `ideal conditions' within the market, Black and Scholes
posit that ``the value of the option will depend only on the value of
the stock and time and on certain variables that are taken to be known
constants,'' \cite{blackscholes1973}. Essentially, Black and Scholes
argue that the option payoff depends on the same underlying source of
randomness as the stock price itself, and thus, it is possible to offset
the randomness in the option's value by appropriately combining it with
the stock, or by modeling the option's price as a function of the
stock's price \cite{blackscholes1973}. When this randomness is
eliminated, the remaining evolution of value would be deterministic.

Black and Scholes obtain a mathematical formulation of this argument,
which leads to a partial differential equation (p.d.e.), known as the
Black---Scholes equation, which the option price must satisfy
\cite{blackscholes1973}. Solving this p.d.e. using It\^{}o's Lemma
yields the following Black--Scholes formula for the price of a European
call option with strike price \(K\) and maturity \(T\) at time \(0\):

\begin{equation}
C(S_0, K, T, r, \sigma)
= S_0\,\Phi(d_1) - K e^{-rT}\,\Phi(d_2),
\label{eq:black_scholes_call_price}
\end{equation}

where,

\begin{equation}
d_1 = \frac{\ln(S_0/K) + (r + \tfrac12 \sigma^2)T}{\sigma\sqrt{T}},
\qquad
d_2 = d_1 - \sigma\sqrt{T}.
\label{eq:black_scholes_d1_d2}
\end{equation}

Additionally, \(r\) is the risk-free interest rate representing the
constant rate of return on an asset that is assumed to have no
uncertainty, and \(\Phi(\cdot)\) denotes the cumulative distribution
function of the standard normal distribution \cite{blackscholes1973}.
This Black--Scholes formula provides a closed-form expression for the
value of a European call option under the assumptions that asset prices
follow geometric Brownian motion and that markets exist under certain
ideal conditions. Despite its simplifying assumptions, the model has
since proven to hold a variety of assets other than European call
options, cementing it as one of the foundational results of modern
financial mathematics \cite{blackscholes1973,pettersdong}.

\subsection{Concluding Remark on the Black-Scholes Framework and its
Connection to the Lognormal
Model}\label{concluding-remark-on-the-black-scholes-framework-and-its-connection-to-the-lognormal-model}

The closed-form solution to the Black--Scholes equation presented in
Section~\ref{sec-exp-bsm} assumes the lognormal distribution of the
terminal stock price \(S(T)\), as established in
Section~\ref{sec-exp-geom}. Under the GBM model, this lognormal
structure implies that the expected payoff of an option has an explicit
analytical expression, which is the Black--Scholes price.

In Section~\ref{sec-sim}, we exploit the same GBM-implied distribution
of \(S(T)\) to compute a hypothetical option's price numerically.
Specifically, we simulate independent realizations of the terminal stock
price under the GBM model, evaluate the corresponding option payoffs,
and approximate their expectation via Monte Carlo averaging. This allows
us to compare the closed-form Black--Scholes price with a
simulation-based estimate of the same theoretical quantity.

\section{Simulation Study: GBM and the Black--Scholes
Formula}\label{sec-sim}

In this section, we illustrate how the GBM modelling assumptions for
stock prices connect to the Black--Scholes option-pricing formula
through a simple simulation experiment. Our goal is to start from the
GBM dynamics introduced earlier and numerically approximate the price of
a hypothetical European call option by averaging simulated payoffs. We
then compare this estimate to the closed-form Black--Scholes price of
this hypothetical option by following the framework presented in
Section~\ref{sec-exp-bsm}, and we expect them to be approximately the
same.

Before we describe the study further, we note that the Black--Scholes
formula and the simulation-based price of an option are both derived
from the same GBM model, but they compute the payoff value in different
ways. The Black--Scholes formula evaluates the expected payoff exactly
using an analytical expression, while the simulation-based price
approximates the same expectation by averaging payoffs across many
simulated GBM outcomes. Comparing the two therefore checks whether the
numerical simulation reproduces the theoretical Black--Scholes price,
with any differences arising from Monte Carlo approximation error.

\subsection{Design of the Toy Option}\label{sec-sim-toy}

We consider a single stock whose price at time \(t\) is modeled by the
GBM process

\begin{equation}
dS(t) = \mu S(t)\,dt + \sigma S(t)\,dB(t),
\label{eq:gbm_sde}
\end{equation}

with the initial price \(S(0) = S_0\). For the simulation, we fix the
following parameter values, with the parameters being defined the same
as in Section~\ref{sec-exp-sde}, Section~\ref{sec-exp-geom}, and
Section~\ref{sec-exp-bsm}.

\begin{equation}
S_0 = 100,\quad K = 100,\quad T = 1,\quad \mu = 0.02,\quad \sigma = 0.20.
\label{eq:simulation_parameters}
\end{equation}

Here, we recall that \(K\) is the strike price, in dollars and \(T\) is
the maturity of the option, in years. The quantity \(\mu\) plays is
essenitally a constant interest rate, and \(\sigma\) is the volatility
parameter from the GBM model that exhibits randomness. We recall from
Section~\ref{sec-exp-bsm} that the payoff of a European call option with
strike \(K\) and maturity \(T\) is,

\begin{equation}
\max(S(T) - K, 0).
\label{eq:call_payoff_T}
\end{equation}

\cite{pettersdong}.

Since \(S(T)\) is lognormally distributed under the GBM model, this
payoff quantity will have a well-defined distribution determined by the
parameters above.

\subsection{Simulating Terminal Prices Under GBM}\label{sec-sim-sim}

From equation \ref{eq:gbm_solution} and by using the notation in
Section~\ref{sec-sim-toy}, we have,

\begin{equation}
S(T) = S(0)\exp\!\left(\big(\mu - \tfrac12 \sigma^2\big)T + \sigma B(T)\right).
\label{eq:gbm_terminal_price}
\end{equation}

In the simulation, we will use the same volatility parameter \(\sigma\)
as in the Black--Scholes formula. To generate a single simulated future
price of the option at maturity time \(T\), termed \(S(T)\), we require
a realization of the Brownian motion \(B(T)\). By definition of Brownian
motion as presented in Section~\ref{sec-exp-brown}, \(B(T)\) is normally
distributed with mean \(0\) and variance \(T\):

\begin{equation}
B(T) \sim N(0,T).
\label{eq:brownian_terminal_distribution_repeat}
\end{equation}

We know, a well-known property of the normal distribution is that if
\(Z \sim N(0,1)\), then for any constant \(a > 0\),

\begin{equation}
aZ \sim N(0, a^{2}).
\label{eq:normal_scaling}
\end{equation}

\cite{blitzstein2019introduction}. Choosing \(a = \sqrt{T}\) gives,

\begin{equation}
\sqrt{T}\,Z \sim N(0,T),
\label{eq:normal_scaling_sqrtT}
\end{equation}

and therefore,

\begin{equation}
B(T) \stackrel{d}{=} \sqrt{T}\,Z,
\qquad Z \sim N(0,1).
\label{eq:brownian_scaling}
\end{equation}

Thus, to generate a realization of \(B(T)\) in our simulation, it
suffices to draw a standard normal random variable \(Z\) and set
\(B(T) = \sqrt{T}\,Z\). Substituting this into the explicit GBM solution
yields the simulation formula,

\begin{equation}
S(T)
= S_0 \exp\!\left( \big(\mu - \tfrac12\sigma^2\big)T
  + \sigma\sqrt{T}\, Z \right),
\qquad Z \sim N(0,1).
\label{eq:gbm_terminal_simulation}
\end{equation}

Repeating this procedure \(N\) times produces \(N\) independent
realizations,

\begin{equation}
S^{(1)}(T),\, S^{(2)}(T),\,\dots,\,S^{(N)}(T).
\label{eq:gbm_terminal_samples}
\end{equation}

of simulated terminal or strike prices of the stock at time \(T\) under
the GBM model.

\subsection{Approximation of the Call Price under GBM}\label{sec-sim-mc}

For each simulated terminal stock price \(S^{(i)}(T)\) from
Section~\ref{sec-sim-sim}, the corresponding payoff or final price of
the option at maturity is,

\begin{equation}
\text{payoff}^{(i)} = \max\big(S^{(i)}(T) - K, 0\big).
\label{eq:call_payoff_simulated}
\end{equation}

Mathematically, it is possible to show that for any option payoff
depending only on the terminal price of a stock, \(S(T)\), the
time--\(0\) price of the option is the discounted expectation of that
payoff under the lognormal distribution of \(S(T)\) generated by the GBM
model \cite{pettersdong}. If we denote the theoretical price of the call
option at \(T=0\) as \(C_0\), then intuitively, the Black--Scholes
framework expresses \(C_0\) as a discounted expectation of the option's
payoff at time \(T\). In other words, since the expected payoff of the
option is computed at time \(T\), we compute its payoff at time \(T=0\)
through `discounting,' \cite{pettersdong}.

Now, since the payoff of a European call option is \(\max(S(T)-K,0)\) as
presented Section~\ref{sec-exp-bsm}, the mathematical representation of
the above intuition yields,

\begin{equation}
C_0 = e^{-\mu T}\,\mathbb{E}\!\left[\max(S(T)-K,0)\right].
\label{eq:call_price_expectation}
\end{equation}

In practice, the expectation in \ref{eq:call_price_expectation} cannot
be evaluated directly, and is instead approximated numerically. We do so
by simulating the terminal stock prices at time \(T\), and then by
replacing the expectation in \ref{eq:call_price_expectation} with a
sample average of \(N\) simulated and independent discounted payoffs, as
demonstrated in Section~\ref{sec-sim-sim}. This yields the GBM estimator
of the time--0 call price, \(\widehat{C}_N\). An estimate of this
approximation's standard error is given by,

\begin{equation}
\text{SE}(\widehat{C}_N)
= \frac{s}{\sqrt{N}},
\label{eq:mc_standard_error}
\end{equation}

where \(s\) is the sample standard deviation of the discounted payoffs.

\subsection{Summary Of Process: Comparing the GBM Simulated Price with
the Analytical Black-Scholes Price}\label{sec-sim-comp}

In this section, we compute the time--0 price of a European call option
under the GBM model in two ways: analytically, using the Black--Scholes
explicit formula, and numerically using a simulated GBM estimate. The
analytical Black--Scholes price, \(C_{\mathrm{BS}}\), is computed using
the closed-form formula in \ref{eq:black_scholes_call_price}, where the
terms \(d_1\) and \(d_2\) are defined in \ref{eq:black_scholes_d1_d2}.
This formula gives the option price directly as a function of ((S\_0, K,
T, r, \sigma)).

The GBM simulation-based price approximates the same time--0 value by
simulating many possible terminal stock prices under GBM. Specifically,
we generate independent draws (S\^{}\{(i)\}(T)) using the terminal
simulation formula in \ref{eq:gbm_terminal_simulation}, compute the
corresponding option payoffs using \ref{eq:call_payoff_simulated}, and
then average the \emph{discounted} payoffs to form the GBM estimator
\(\widehat{C}_N\). The standard error of this Monte Carlo estimate is
summarized by \ref{eq:mc_standard_error}.

Because both methods are derived from the same GBM assumptions, they are
estimating the same theoretical price. So, we expect that with
increasing \(N\), the value of \(\widehat{C}_N\) converges to the
analytical value of \(C_{\mathrm{BS}}\). Any differences between
\(C_{\mathrm{BS}}\) and \(\widehat{C}_N\) are expected to arise due to
sampling error.

\subsection{Implementation}\label{sec-sim-imp}

In this sub-section, we briefly describe the reproducible Python code
used to carry out the simulation study introduced in
Section~\ref{sec-sim-sim}, Section~\ref{sec-sim-mc}, and
Section~\ref{sec-sim-comp}. The full implementation of the code is
provided in a reproducible Google Colab notebook
\cite{rashha2025simulationstudy}, which will be referring to throughout
this sub-section.

The goal of the implementation is to consider a hypothetical European
call option with pre-specified parameters as defined in
Section~\ref{sec-sim-toy}, generate simulated terminal stock prices
under the GBM model, compute the corresponding payoffs for the
hypotehtical option, and compare the resulting GBM estimates with the
analytical Black--Scholes value of the option's payoff. The code is
organized into three key components: (1) a function that evaluates the
Black--Scholes closed-form formula to compute the payoff of the
hypothetical option, (2) a function that simulates terminal stock prices
using the explicit GBM solution, and (3) an estimator function that
averages the simulated payoffs, discounts it to time--0, and thus
computes the GBM simulated payoff of the option.

\subsubsection{Black--Scholes Pricing Function}\label{sec-sim-imp-1}

The first component of the code is a function that evaluates the
theoretical price of a European call option using the Black--Scholes
formula presented in in Section~\ref{sec-exp-bsm}. Given the parameter
tuple \((S_0, K, T, r, \sigma)\), the pricing function computes the
quantities \(d_1\) and \(d_2\), and returns the corresponding
closed-form option value at time--0.

\subsubsection{GBM Simulator Function}\label{sec-sim-imp-2}

To simulate realizations of the terminal stock price \(S(T)\) under the
GBM model, we use the explicit solution to the stochastic differential
equation presented in Section~\ref{sec-exp-geom}. Under this model, the
terminal price is given by equation \ref{eq:gbm_terminal_simulation}.
Independent draws of the normally distributed random variable \(Z\)
yield independent realizations of \(S(T)\), as explained in
Section~\ref{sec-sim-sim}, allowing for an approximation of expectations
involving the terminal stock price.

\subsubsection{GBM Estimator Function}\label{sec-sim-imp-3}

Using simulated terminal prices \(S^{(i)}(T)\), the GBM estimator of the
European call option price is constructed by averaging discounted
payoffs across \(N\) independent simulation paths. The estimator is
given by

\begin{equation}
\widehat{C}_N
= e^{-rT}\,\frac{1}{N}\sum_{i=1}^{N} \max\!\left(S^{(i)}(T) - K,\, 0\right).
\label{eq:call_payoff_simulated}
\end{equation}

\cite{pettersdong}, \cite{blackscholes1973}.

In order to ensure consistency with the Black--Scholes framework, the
drift parameter \(\mu\) used in the simulation is set equal to the
risk-free rate \(r\) since they both capture the deterministic rate of
change over time. The variability of this estimator is quantified by its
standard error, which this function computes, along with the following
95\% confidence interval,

\begin{equation}
\widehat{C}_N \pm 1.96 \times \mathrm{SE}\!\left(\widehat{C}_N\right).
\label{eq:mc_confidence_interval}
\end{equation}

\subsection{Simulation Results and Analysis}\label{sec-sim-res}

Finally, we compare the Black-Scholes analytical price and the GBM
simulated price of the payoff of our hypothtical European call option
based on the results obtained from the implementation of this study, as
described in previous sections.

Using the code in \cite{rashha2025simulationstudy}, the Black-Scholes
price of the option is computed as,

\begin{equation}
C_{\mathrm{BS}} = 8.9160.
\label{eq:black_scholes_numeric}
\end{equation}

We treat this value serves as a reference point for assessing the
accuracy of the simulation-based estimator using GBM price paths of the
stocks. Table \ref{tab:mcresults} summarizes the estimates of the option
prices for a varying number of simulated GBM paths that were tested:

\begin{table}[h!]
\centering
\begin{tabular}{r|c|c|c|c|c}
\hline
$N_{\text{paths}}$ & $\widehat{C}_N$ & Std.\ Error & 95\% CI Lower & 95\% CI Upper & $\lvert \widehat{C}_N - C_{\mathrm{BS}} \rvert$ \\
\hline
1{,}000   & 8.4335 & 0.4229 & 7.6046 & 9.2623 & 0.4826 \\
5{,}000   & 9.0358 & 0.1926 & 8.6584 & 9.4132 & 0.1197 \\
20{,}000  & 9.0621 & 0.0984 & 8.8692 & 9.2549 & 0.1460 \\
100{,}000 & 8.9361 & 0.0437 & 8.8505 & 9.0216 & 0.0200 \\
\hline
\end{tabular}
\caption{GBM Estimates of the European Call Price using Different Numbers of Simulated Paths.}
\label{tab:mcresults}
\end{table}

From the results above, we observe that as the number of simulated paths
\(N\) increases, the GBM price estimate converges toward the
Black--Scholes benchmark price. For instance, the estimate based on
\(1{,}000\) paths differs from \(C_{\mathrm{BS}}\) by approximately
\(0.48\) units, whereas the estimate based on \(100{,}000\) paths
differs by only \(0.02\) units. This convergent behavior is what we
expect from this simulation, since the GBM estimator is essentially a
sample average of payoffs computed from many GBM-simulated terminal
stock prices, and therefore converges to the true expectation \(C_0\) as
\(N\) increases. This is also consistent with the theory of the
Black-Scholes Pricing Framework, which assumes GBM as the underlying
distribution of the assets when computing the analytical price of the
options \cite{pettersdong}, \cite{blackscholes1973}.

\section{Application of Geometric Brownian Motion and The Black-Scholes
Framework: Real Markets}\label{sec-app}

\subsection{Motivations}\label{sec-app-mot}

In this section, we will apply the Geometric Brownian Motion (GBM)
framework and the Black--Scholes option pricing model to real financial
data in order to assess how well the theoretical assumptions examined
earlier align with observed market behavior. Because publicly available
data on option payoffs are limited, this application focuses on modeling
stock prices, for which high-frequency historical data are readily
available. This approach is standard in empirical finance, since GBM and
the Black-Scholes formula have been shown to hold for a variety of
assets other than European call options \cite{pettersdong}.

Using daily closing values of the S\&P\textasciitilde500 index, we
estimate the parameters of a GBM model from historical data, examine
whether empirical log returns are consistent with the normality
assumption implied by the model, and use the fitted parameters to price
a hypothetical European call option. By comparing GBM--based price
estimates computed following the framework in Section~\ref{sec-sim-imp}
with the corresponding Black--Scholes closed-form value, this
application illustrates how the GBM model performs when calibrated to
real market data and highlights the extent to which theoretical option
pricing results are supported in practice.

All code for this section has been written in Python, is fully
reproducible, and can be found in \cite{rashha2025applicationcode}.

\subsection{The Dataset}\label{sec-app-dat}

For the application, we use a dataset of the daily
S\&P\textasciitilde500 index values obtained from the publicly available
\texttt{xLSTM-TS} repository maintained by Lopez Gil
\cite{gil2024xlstmtsdata}.

\begin{Shaded}
\begin{Highlighting}[]
\NormalTok{sp500\_df }\OtherTok{\textless{}{-}} \FunctionTok{read\_csv}\NormalTok{(}\StringTok{"/Users/nahianrashha/Desktop/sp500\_daily.csv"}\NormalTok{)}
\end{Highlighting}
\end{Shaded}

\begin{verbatim}
Rows: 6037 Columns: 8
-- Column specification --------------------------------------------------------
Delimiter: ","
dbl  (7): Open, High, Low, Close, Volume, Dividends, Stock Splits
dttm (1): Date

i Use `spec()` to retrieve the full column specification for this data.
i Specify the column types or set `show_col_types = FALSE` to quiet this message.
\end{verbatim}

\begin{Shaded}
\begin{Highlighting}[]
\FunctionTok{glimpse}\NormalTok{(sp500\_df)}
\end{Highlighting}
\end{Shaded}

\begin{verbatim}
Rows: 6,037
Columns: 8
$ Date           <dttm> 2000-01-03 05:00:00, 2000-01-04 05:00:00, 2000-01-05 0~
$ Open           <dbl> 1469.25, 1455.22, 1399.42, 1402.11, 1403.45, 1441.47, 1~
$ High           <dbl> 1478.00, 1455.22, 1413.27, 1411.90, 1441.47, 1464.36, 1~
$ Low            <dbl> 1438.36, 1397.43, 1377.68, 1392.10, 1400.73, 1441.47, 1~
$ Close          <dbl> 1455.22, 1399.42, 1402.11, 1403.45, 1441.47, 1457.60, 1~
$ Volume         <dbl> 931800000, 1009000000, 1085500000, 1092300000, 12252000~
$ Dividends      <dbl> 0, 0, 0, 0, 0, 0, 0, 0, 0, 0, 0, 0, 0, 0, 0, 0, 0, 0, 0~
$ `Stock Splits` <dbl> 0, 0, 0, 0, 0, 0, 0, 0, 0, 0, 0, 0, 0, 0, 0, 0, 0, 0, 0~
\end{verbatim}

The dataset spans the period from March 1,\textasciitilde2000 to
December 29,\textasciitilde2023 and contains daily observations of the
S\&P\textasciitilde500 index. The dataset is stored locally as
\texttt{sp500\_daily.csv}, has 6037 observations, each representing a
single trading day within the described period, and contains eight
columns: \texttt{Date}, \texttt{Open}, \texttt{High}, \texttt{Low},
\texttt{Close}, \texttt{Volume}, \texttt{Dividends}, and
\texttt{Stock Splits}.

The \texttt{Close} column records the closing level or prices of the
S\&P\textasciitilde500 index on each trading day and serves as the price
process \(\{S(t)\}\) used throughout the analysis. Because the data are
recorded at a daily frequency, we will set the time increment used for
computing log returns to \(\Delta t = 1/252\), reflecting the
conventional assumption of approximately 252 trading days per year in
U.S.~equity markets.

In the following section, we use the daily closing prices from this
dataset to estimate the parameters of a Geometric Brownian Motion model
and to assess whether the empirical distribution of log returns is
consistent with the normality assumption implied by the GBM framework.

\subsection{Estimation of the GBM Model from Market
Data}\label{sec-app-est}

Using the daily closing prices of the S\&P500 index described above, we
model the price process \(\{S_t\}\) as a Geometric Brownian Motion,
following the theoretical framework introduced in
Section~\ref{sec-exp-geom}. The estimation procedure is based on
continuously compounded log returns, \begin{equation}
r_t = \log\!\left(\frac{S_t}{S_{t-1}}\right),
\label{eq:log_returns}
\end{equation}

which, under the GBM assumption, are independent and normally
distributed with parameters determined by the drift \(\mu\) and
volatility \(\sigma\) \cite{pettersdong}.

In the application code, the dataset is first ordered chronologically,
rows with missing values are omitted, and then, the \texttt{Close}
column is extracted as the observed price series. Daily log returns are
then computed using \ref{eq:log_returns}. Because the data are observed
at a daily frequency, the time increment is fixed at \begin{equation}
\Delta t = \frac{1}{252},
\label{eq:dt}
\end{equation} corresponding to the standard convention of 252 trading
days per year.

We then estimate the GBM parameters using sample moments of the daily
log returns. Specifically, the sample variance of log returns is used to
estimate the volatility parameter \(\sigma\), while the drift parameter
\(\mu\) is obtained by rescaling the sample mean of log returns and
accounting for the Itô correction implied by the GBM solution, as
discussed Section~\ref{sec-exp-geom}. This moment-based estimation
approach follows from the closed-form distribution of GBM log returns.

\subsection{Theoretical Basis for Parameter
Estimation}\label{sec-app-theo}

The estimation strategy used in this application follows from the
analytical properties of the Geometric Brownian Motion (GBM) model
introduced earlier in the report. Under GBM, the asset price process
satisfies \begin{equation}
dS_t = \mu S_t \, dt + \sigma S_t \, dB_t,
\label{eq:gbm_sde}
\end{equation} where \(\mu\) is the drift parameter and \(\sigma\) is
the volatility parameter.

As shown in Section Section~\ref{sec-exp-geom}, from
\ref{eq:gbm_solution}, this stochastic differential equation has the
exact closed-form solution \begin{equation}
S_t = S_0 \exp\!\left( (\mu - \tfrac{1}{2}\sigma^2)t + \sigma B_t \right),
\label{eq:gbm_solution_2}
\end{equation}

\cite{pettersdong}.

Now, taking logarithms of price ratios over a fixed time increment
\(\Delta t\) yields the log return \begin{equation}
\log\!\left(\frac{S_t}{S_{t-\Delta t}}\right)
= (\mu - \tfrac{1}{2}\sigma^2)\Delta t + \sigma (B_t - B_{t-\Delta t}).
\label{eq:log_return_expression}
\end{equation}

From Section~\ref{sec-exp-brown} we know that Brownian motion has
independent increments satisfying \begin{equation}
B_t - B_{t-\Delta t} \sim \mathcal{N}(0, \Delta t),
\label{eq:brownian_increment}
\end{equation}

\cite{pettersdong}. From this, it follows that the log returns are
normally distributed,

\begin{equation}
r_t := \log\!\left(\frac{S_t}{S_{t-1}}\right)
\sim \mathcal{N}\!\left( (\mu - \tfrac{1}{2}\sigma^2)\Delta t,\; \sigma^2 \Delta t \right).
\label{eq:log_return_distribution}
\end{equation}

Thus, the population mean and variance of log returns would staisfy the
following,

\begin{equation}
\mathbb{E}[r_t] = (\mu - \tfrac{1}{2}\sigma^2)\Delta t,
\label{eq:gbm_mean}
\end{equation} and \begin{equation}
\mathrm{Var}(r_t) = \sigma^2 \Delta t.
\label{eq:gbm_variance}
\end{equation}

These relationships provide the necessary theoretical justification for
the estimation approach used in our application. Under the GBM
assumptions, we expect that the sample mean and sample variance of
observed log returns would converge to \ref{eq:gbm_mean} and
\ref{eq:gbm_variance}, respectively. The volatility parameter \(\sigma\)
can therefore be estimated from the sample variance, while the drift
parameter \(\mu\) is recovered by rearranging the mean relationship.
Using the estimated parameters, the implied daily mean and variance of
log returns would then be,

\begin{equation}
\mathbb{E}[r_t] = (\hat{\mu} - \tfrac{1}{2}\hat{\sigma}^2)\Delta t,
\label{eq:gbm_mean_2}
\end{equation} and \begin{equation}
\mathrm{Var}(r_t) = \hat{\sigma}^2 \Delta t.
\label{eq:gbm_var_2}
\end{equation}

In the application, these quantities are used to compare the empirical
distribution of observed log returns to the normal distribution implied
by the fitted GBM model to the data \cite{rashha2025applicationcode}. We
coduct this comparison by visually inspecting the distribution of the
simulated of the log-returns and comparing it to the normal distribution
to evaluate whether the normality assumption underlying GBM is
reasonable for the S\&P\textasciitilde500 data.

\subsection{Hypothetical Option Pricing under
Black--Scholes}\label{sec-app-hyp}

With the volatility parameter estimated from historical data, we apply
the Black--Scholes framework introduced in \ref{sec-exp-bsm} to price a
hypothetical European call option specified by a maturity \(T\), strike
price \(K\), and a constant risk-free interest rate \(r\).

The drift used in simulation is set equal to the risk-free rate \(r\),
while volatility is fixed at the estimated value \(\hat{\sigma}\). Two
pricing approaches are implemented. First, the Black--Scholes
closed-form solution is evaluated using the fitted parameters. Second, a
GBM simulated estimate is obtained by simulating terminal prices from
the GBM solution and obtaining the discounted average payoff. This
comparison mirrors the simulation-based pricing framework developed in
\ref{sec-sim-imp} and could aid in illustrating how the analytical and
numerical approaches relate when calibrated to real market data.

\subsection{Implementation Overview}\label{sec-app-imp}

The application code performs the following steps:

\begin{enumerate}
\def\labelenumi{\arabic{enumi}.}
\tightlist
\item
  Loads daily S\&P\textasciitilde500 index data and extracts the closing
  price series.
\item
  Computes daily log returns and sets the time increment according to
  equation \ref{eq:dt}.
\item
  Estimates the GBM drift and volatility using sample moments of log
  returns.
\item
  Evaluates the normality assumption implied by the GBM model using
  graphical diagnostics.
\item
  Prices a hypothetical European call option using both the
  Black--Scholes formula and the GBM estimated volatility parameter.
\end{enumerate}

\cite{rashha2025applicationcode}

In the next section, we examine the results obtained from this
application example.

\subsection{Application Parameters and Results}\label{sec-app-res}

\subsubsection{Empirical Log Returns and GBM
Fit}\label{empirical-log-returns-and-gbm-fit}

\begin{figure}

\centering{

\includegraphics[width=0.7\linewidth,height=\textheight,keepaspectratio]{hist.png}

}

\caption{\label{fig-hist}Histogram of daily S\&P\textasciitilde500 log
returns overlaid on GBM-implied normal distribution.}

\end{figure}%

Figure~\ref{fig-hist} compares the empirical distribution of daily log
returns for the S\&P\textasciitilde500 to the normal distribution
implied by the fitted GBM model. We observe that while the fitted
distribution of log returns captures the center of the normal
distribution reasonably well, the empirical log returns exhibit
substantial departures from normality.

The estimated sample moments of daily log returns are, \begin{equation}
\hat{m} = 0.00053742, \qquad \hat{v} = 3.58197760 \times 10^{-4},
\label{eq:sample_moments}
\end{equation}

Using equations \ref{eq:gbm_mean_2} and \ref{eq:gbm_var_2} and inputting
the above values of the sample moments, we obtain the following
estimated GBM parameters for drift and volatility respectively:

\begin{equation}
\hat{\mu} = 0.1806, \qquad \hat{\sigma} = 0.3004.
\label{eq:gbm_params}
\end{equation}

The computed higher-order moments indicate pronounced departures from
normality. The empirical skewness and excess kurtosis are,

\begin{equation}
\text{Skewness} = 3.4924, \qquad \text{Excess kurtosis} = 156.1138.
\label{eq:higher_moments}
\end{equation}

These results likely capture the well-known heavy-tailed nature real of
equity markets, and a key limitation of the GBM assumption
\cite{article}, \cite{pettersdong}.

\subsubsection{GBM Forecast
Distribution}\label{gbm-forecast-distribution}

\begin{figure}[H]

\centering{

\includegraphics[width=0.7\linewidth,height=\textheight,keepaspectratio]{gbm_paths.png}

}

\caption{\label{fig-paths}Sample of Simulated GBM Price Paths fitted on
the S\&P500.}

\end{figure}%

Using the above fitted GBM parameters in \ref{eq:gbm_params}, we
generate a one-year-ahead distribution for the S\&P\textasciitilde500
index. Starting from the current index level or the last observed
closing value of the index, \begin{equation}
S_0 = 4769.83,
\label{eq:s0}
\end{equation} the simulated GBM forecast distribution yields the
following quantiles: \begin{equation}
S_{0.05}(T) \approx 3333.80, \quad
S_{0.50}(T) \approx 5440.97, \quad
S_{0.95}(T) \approx 8949.06.
\label{eq:forecast_quantiles}
\end{equation}

Figure~\ref{fig-paths} illustrates a subset of simulated GBM index price
paths over the one-year horizon. The wide dispersion of trajectories
reflects the relatively large estimated volatility and likely indicates
the uncertainty inherent in long-term equity forecasts under the GBM
model \cite{pettersdong}.

\label{fig:gbm_paths}

\subsection{Option Pricing Results}\label{option-pricing-results}

Using the historical volatility estimate \(\hat{\sigma}\), a
hypothetical European call option is priced under the Black--Scholes
framework. The option parameters are \begin{equation}
T = 1.0, \qquad r = 0.04, \qquad K = 1.05 S_0 = 5008.32.
\label{eq:option_params}
\end{equation}

The Black--Scholes closed-form price is \begin{equation}
C_{\mathrm{BS}} = 551.2667.
\label{eq:bs_price}
\end{equation}

A GBM estimate based on \(N = 200{,}000\) simulated GBM paths yields
\begin{equation}
\hat{C}_{\mathrm{MC}} = 553.1366,
\label{eq:mc_price}
\end{equation} with standard error \begin{equation}
\mathrm{SE} = 2.2193,
\label{eq:mc_se}
\end{equation} and corresponding 95\% confidence interval
\begin{equation}
[548.7868,\;557.4863].
\label{eq:mc_ci}
\end{equation}

The absolute difference between the Monte Carlo estimate and the
Black--Scholes price is \begin{equation}
\lvert \hat{C}_{\mathrm{MC}} - C_{\mathrm{BS}} \rvert = 1.8699.
\label{eq:price_diff}
\end{equation}

The relatively narrow absolute difference between the GBM simulated
price of the hypothetical option and the analytical Black-Scholes
framework reaffirms our discussion in Section~\ref{sec-sim-res}, where
we saw that for large \(N\), we expect the GBM simulated price of an
asset to converge to the Black-Scholes one. This agreement shows that,
conditional on the fitted volatility parameter, the simulation-based
pricing procedure is numerically consistent with the Black--Scholes
formula.

Overall, this section illustrates how the GBM framework operates when
calibrated to real market data. Historical S\&P\textasciitilde500
returns provide a natural basis for estimating the model's first two
moments, which in turn determine the volatility parameter used for
forecasting and option pricing. While the fitted GBM model produces
coherent simulations and yields fairly consistent Black--Scholes and
GBM-simulated option prices, diagnostic analysis of log returns reveals
substantial departures of its distribution from the normality assumption
underlying GBM. As a result, the application highlights both the
practical usefulness of GBM as an approximation of the behavior of asset
prices and the importance of interpreting its outputs with caution when
modeling real markets.

\section{Conclusion}\label{conclusion}

In this paper, we examined geometric Brownian motion (GBM) as a
foundational model for asset prices in equity markets, and its role in
developing the Black--Scholes framework for pricing options. In
Section~\ref{sec-exp}, we explored how GBM arises from Brownian motion,
its stochastic properties, including how it yields lognormally
distributed prices for assets, and its closed-form solution for terminal
prices of assets. Finally, we observed GBM's connection to the
Black--Scholes formula for options under ideal market conditions.

In Section~\ref{sec-sim}, the simulation study served as a bridge
between theory and computation. By simulating terminal prices of stocks
directly using the closed-form solution of the GBM stochastic
differential equation, the study demonstrated how it is possible to
approximate option values as discounted expected payoffs under GBM. The
close agreement between the GBM estimates and the Black--Scholes
closed-form price in the toy example illustrated the consistency of the
models and solidified the theoretical connection between two
foundational contributions to modern-day financial mathematics by Paul
Samuelson, Fischer Black, and Merton Scholes \cite{samuelson1965},
\cite{blackscholes1973}.

The final application to real S\&P 500 data highlighted both the
practical utility and the limitations of the GBM model. Calibrating GBM
parameters using historical daily log returns allowed us to construct
many possible future index paths, and the price a hypothetical European
call option using the Black--Scholes framework. While the fitted model
matched the empirical mean and variance of log returns by construction,
diagnostic plots revealed substantial deviations from normality in
higher-order moments, reflecting skewness and heavy tails commonly
commonly observed in equity markets \cite{article}. However, conditional
on the estimated volatility parameter, the Black--Scholes price and its
GBM approximation of the option price remained numerically consistent,
reinforcing the interpretation of both GBM and Black--Scholes as
model-based pricing tools and the notion of exercising caution when
using them to capture real market dynamics .

\section{Appendix: Covariance of Brownian Motion}\label{sec-appix}

Let \(\{B(t) : t \ge 0\}\) be a standard Brownian motion. We derive the
covariance function \(\mathrm{Cov}(B(s), B(t))\) using the defining
properties of Brownian motion---independent increments and Gaussian
increments with variance equal to the time increment---as in Petters and
Dong \cite{pettersdong}.

Fix \(0 \le s \le t\). Then we can write \begin{equation}
B(t) = B(s) + \big( B(t) - B(s) \big).
\label{eq:bm_decomposition}
\end{equation}

By the independent-increments property of Brownian motion, \(B(s)\) and
\(B(t) - B(s)\) are independent random variables. Therefore,
\begin{equation}
\mathrm{Cov}\big( B(s), B(t) \big)
= \mathrm{Cov}\big( B(s),\, B(s) + (B(t) - B(s)) \big).
\label{eq:cov_step1}
\end{equation}

Using the bilinearity of covariance, \begin{equation}
\mathrm{Cov}\big( B(s),\, B(s) + (B(t) - B(s)) \big)
= \mathrm{Cov}\big( B(s), B(s) \big)
+ \mathrm{Cov}\big( B(s), B(t) - B(s) \big).
\label{eq:cov_bilinear}
\end{equation}

Since \(B(s)\) and \(B(t) - B(s)\) are independent, their covariance is
zero: \begin{equation}
\mathrm{Cov}\big( B(s), B(t) - B(s) \big) = 0.
\label{eq:cov_zero}
\end{equation}

Hence, \begin{equation}
\mathrm{Cov}\big( B(s), B(t) \big)
= \mathrm{Cov}\big( B(s), B(s) \big)
= \mathrm{Var}\big( B(s) \big).
\label{eq:cov_equals_var}
\end{equation}

From the definition of Brownian motion, we know that \begin{equation}
B(s) - B(0) \sim \mathcal{N}(0, s),
\label{eq:bm_variance}
\end{equation} which implies \begin{equation}
\mathrm{Var}\big( B(s) \big) = s.
\label{eq:var_bs}
\end{equation}

Therefore, for \(0 \le s \le t\), \begin{equation}
\mathrm{Cov}(B(s), B(t)) = s.
\label{eq:cov_st_ordered}
\end{equation}

By symmetry of covariance, \begin{equation}
\mathrm{Cov}(B(s), B(t)) = \mathrm{Cov}(B(t), B(s)),
\label{eq:cov_symmetry}
\end{equation} and thus, for arbitrary \(s,t \ge 0\), \begin{equation}
\mathrm{Cov}(B(s), B(t)) = \min(s,t).
\label{eq:cov_min}
\end{equation}

This proves the result.


\bibliography{bibliography.bib}



\end{document}
